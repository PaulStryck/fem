\documentclass[11pt,a4paper]{article}

\usepackage[utf8]{inputenc}
\usepackage[english]{babel}
\usepackage[left=2cm,right=2cm,top=2cm,bottom=2cm]{geometry}

\usepackage{csquotes}
\usepackage{graphicx}

\usepackage{amsmath}
\usepackage{amssymb}
\usepackage{amsthm}

\usepackage{mathrsfs}
\usepackage{mathtools}

\usepackage{color}
\usepackage{epsfig}
\usepackage{array}
\usepackage{multicol}
\usepackage{tikz}
\usepackage{listings}
\usepackage{minted}
\usepackage{mdframed}

\setlength{\parindent}{0em}
\setlength{\parskip}{0.5em}
% \textwidth 6.5in
% \textheight 9.in
% \oddsidemargin 0in
% \headheight 0in



\definecolor{codegreen}{rgb}{0,0.6,0}
\definecolor{codegray}{rgb}{0.5,0.5,0.5}
\definecolor{backcolour}{rgb}{0.95,0.95,0.95}

\usepackage[hidelinks]{hyperref}
\hypersetup{
    colorlinks=false, %set true if you want colored links
    linktoc=all,      %set to all if you want both sections & subsections linked
}


\lstset{ %
  language=python,                % choose the language of the code
  basicstyle=\footnotesize,       % the size of the fonts that are used for the code
  numbers=left,                   % where to put the line-numbers
  numberstyle=\footnotesize,      % the size of the fonts that are used for the line-numbers
  stepnumber=1,                   % the step between two line-numbers. If it is 1 each line will be numbered
  numbersep=5pt,                  % how far the line-numbers are from the code
  backgroundcolor=\color{white},  % choose the background color. You must add \usepackage{color}
  showspaces=false,               % show spaces adding particular underscores
  showstringspaces=false,         % underline spaces within strings
  showtabs=false,                 % show tabs within strings adding particular underscores
  frame=single,                   % adds a frame around the code
  tabsize=2,                      % sets default tabsize to 2 spaces
  captionpos=b,                   % sets the caption-position to bottom
  breaklines=true,                % sets automatic line breaking
  breakatwhitespace=false,        % sets if automatic breaks should only happen at whitespace
  escapeinside={\%*}{*)}          % if you want to add a comment within your code
}

\usemintedstyle{vs}


\begin{document}


\usetikzlibrary{positioning}
\tikzset{every picture/.style={line width=0.75pt}}

\pagestyle{plain}

\begin{multicols}{2}
  \begin{flushleft}
    MAT360 \\
    Prof. Jan Martin Nordbotten\\
    \underline{University of Bergen}
  \end{flushleft}
  \vfill\null
  \columnbreak

  \begin{flushright}
    \includegraphics[height=2cm]{assets/uib.logo.png}
  \end{flushright}
\end{multicols}

\begin{center}
\textbf{\large Exemplary FEM implementations and convergence analysis}\\
Paul Stryck\\
\end{center}
\rule{\linewidth}{0.1mm}



\begin{abstract}
    \noindent
    This report is meant to outline the thought process behind the implementation of an exemplary fem code,
    solving the Laplace equation on the unit square with both Dirichlet and von Neumann boundary conditions.
    The implementation will work with structured and unstructured grids, which are generated on the unit square.
    It is then extended to process grids on arbitrary 2d geometries in the {\it .msh} file format from {\it gmsh}.
    The convergence rates will be numerically obtained for structured and unstructured grids on the unit square
    for example problems where the analytical solution is known.
\end{abstract}

\begin{multicols}{2}

  \subsection*{Theory}
  \subsection*{Grid Generation}

\end{multicols}

\end{document}
